\begin{flushleft}
\sigle{ENSIMAG -- Grenoble INP -- UGA} \hfill \sigle{Year} 2023-2024 \\
Introduction to Statistical Learning and Applications \\
Pedro L. C. Rodrigues   \hfill  \texttt{pedro.rodrigues@inria.fr} \\
Alexandre Wendling \hfill \texttt{alexandre.wendling@grenoble-inp.org} \\

\HRuleTop\\

\section*{\faExclamationTriangle~General guidelines for TPs}

Each team shall upload its report on Teide before the deadline indicated at the course website. Please \textbf{include the name of all members of the team} on top of your report. 

The report should contain graphical representations. For each graph, axis names should be provided as well as a legend when it is appropriate. Figures should be explained by a few sentences in the text. Answer to the \textbf{questions in order and refer to the question number in your report}. Computations and graphics have to be performed with \texttt{R}.

The report should be written using the \texttt{Rmarkdown} format. This is a file format that allows users to format documents containing text and \texttt{R} instructions. You should include all of the \texttt{R} instructions that you have used in the \texttt{rmd} document so that it may be possible to replicate your results. From your \texttt{rmd} file, you are asked to generate an \texttt{html} file for the final report. In Teide, you are asked to submit both the \texttt{rmd} and the \texttt{html} files. In the \texttt{html} file, you should limit the displayed \texttt{R} code to the most important instructions.

\HRuleTop\\
\begin{center}
\Large{TP 1: Analysis of prostate cancer data}
\end{center}
\HRuleBottom
\end{flushleft}
